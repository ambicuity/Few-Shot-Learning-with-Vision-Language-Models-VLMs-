\section{Methodology}

We propose \textbf{Dual-Alignment Prompt Tuning (DAPT)}, a framework designed for robust vocabulary-free few-shot adaptation. 
DAPT addresses the challenge of aligning visual features with semantic space when class names are unavailable or noisy.

\subsection{Overview}
Our architecture consists of two frozen encoders (image and text) from a pre-trained VLM (e.g., CLIP) and a lightweight trainable adapter module. 
Unlike standard CoOp which optimizes prompts solely against a fixed set of class names, DAPT introduces a \textit{Visual Prototype Branch} that learns class centroids directly in the embedding space, independent of textual labels.

\subsection{Visual Prototype Branch}
Let $f_i$ be the image feature for a support sample $x_i$ belonging to class $y$. 
We compute a prototype $p_c$ for each class $c$ as the mean of its support features. 
To refine these prototypes, we employ a non-parametric attention mechanism that weighs outlier samples less heavily, ensuring robustness against label noise—a critical factor in real-world industrial datasets.

\subsection{Learnable Prompt Alignment}
Simultaneously, we maintain a set of learnable prompt vectors $v \in \mathbb{R}^{L \times D}$. 
Instead of projecting these directly to class logits, we align them with the Visual Prototypes via a contrastive loss $\mathcal{L}_{align}$.
This constraint forces the prompt embeddings to capture the visual structure of the unseen classes, effectively "naming" the unnamed clusters in the semantic space.

\subsection{Inference}
During inference, the classification score is a fused metric of the cosine similarity to the learned text prompts and the distance to the visual prototypes:
\begin{equation}
    S(x) = \alpha \cdot \text{sim}(E_{img}(x), g(v)) + (1-\alpha) \cdot \text{sim}(E_{img}(x), p_c)
\end{equation}
where $\alpha$ is a hyperparameter balancing the two streams.
Unlike naive ensembling, DAPT enforces alignment during training, ensuring prompt embeddings and visual prototypes co-evolve toward a shared semantic geometry rather than being independently optimized.
This approach allows DAPT to generalize well (via prompts) while maintaining high specificity (via prototypes).

\begin{itemize}
    \item \textbf{Contribution 1:} A vocabulary-free adaptation mechanism that functions without explicit class names.
    \item \textbf{Contribution 2:} A dual-path comparison strategy that outperforms both pure prompt tuning and pure prototype methods.
    \item \textbf{Contribution 3:} Superior training efficiency (<1 minute on standard GPUs) and robustness to domain shift.
\end{itemize}
